\documentclass[10pt]{beamer}

\mode<presentation>
{
  \usetheme{Darmstadt}      % or try Darmstadt, Madrid, Warsaw, ...
  \usecolortheme{rose} % or try albatross, beaver, crane, ...
  \usefonttheme{structurebold}  % or try serif, structurebold, ...
  \setbeamertemplate{navigation symbols}{}
  \setbeamertemplate{caption}[numbered]
}

\usepackage[english]{babel}
\usepackage[utf8x]{inputenc}
\usepackage{mathtools}
\usepackage{bbm}

% \usepackage{amsfonts}
% \usepackage{amsmath}
% \usepackage{amssymb}

\title[Life Insurance Mathematics]{Exam Presentation \\Life Insurance Mathematics}
\author{Sami Zweidler \\ Actuary CLP at Zurich Insurance}
\institute{ETH Zurich}
\date{20/01/2026}

\AtBeginSection[]{
  \begin{frame}[plain]
    \centering
    \vfill
    {\Huge\insertsection}
    \vfill
  \end{frame}
}


\begin{document}

\begin{frame}
  \titlepage
\end{frame}


\begin{frame}[plain]
	\tableofcontents
\end{frame}
% Uncomment these lines for an automatically generated outline.
%\begin{frame}{Outline}
%  \tableofcontents
%\end{frame}

% \section{Task 1: Markov Model}
\section{Task 1: Markov Model}
\begin{frame}{Markov Chains}
	What is a Markov Chain?\\
	It is a Stochastic model that describes sequence of transitions/ possible events in which the probability of each event depends only on the state attained in the previous event.
	\begin{figure}
		\includegraphics[width=0.5\textwidth]{Plots/Example_Markov.png}
	\end{figure}
\end{frame}


\begin{frame}{Markow Chains}
    \begin{block}{Definition}
		$(X_t)_{t\in \mathbb{N}}: (\Omega, \mathcal{A}, \mathbb{P}) \to \mathcal{S} = \{1,2,3, \ldots\}$, is called a \textbf{Markov chain} if and only if,
		\begin{equation*}
			\mathbb{P}[X_{t_{m+1}} = i_{m+1} | X_{t_1}=i_1, \dots, X_{t_m}=i_m] = \mathbb{P}[X_{t_{m+1}} = i_{m+1} | X_{t_m}=i_m]
		\end{equation*}
		for $t_1 < t_2 < \dots < t_m < t_{m+1}$ and $i_1, i_2, \ldots, i_{m+1} \in \mathcal{S}$.
	\end{block}
	We say that such a stochastic process $(X_t)_{t\in \mathbb{N}}$ has no memory.
\end{frame}


\begin{frame}{Markov Chains}
    \begin{block}{Chapman-Kolmogorov Theorem}
		Let $p_{ij}(s,t) = P(X_t = j | X_s = i)$ be the transition probabilities of a Markov chain. Then, for any $0 \leq s < u < t$,
		\begin{equation*}
			p_{ij}(s,t) = \sum_{k} p_{ik}(s,u) p_{kj}(u,t).
		\end{equation*}
		Or written in matrix form, $P(s,t) = P(s,u) P(u,t)$.
	\end{block}
	Idea: What is the probbility of being in state j at time t, given that at time s we are in state i?
\end{frame}

\begin{frame}{Markov Chains}
	\begin{block}{Proof}

		\begin{align*}
			p_{ij}(s,t) &= \mathbb{P}[X_t = j | X_s = i] = \mathbb{P}[X_t = j \cap \bigcup\limits_{k \in \mathcal{S}}\{X_u = k\} | X_s=i] \\
			&= \sum\limits_{k \in \mathcal{S}}\mathbb{P}\left[X_t = j, X_u = k | X_s=i\right] \\
			&= \sum\limits_{k \in \mathcal{S}} \frac{\mathbb{P}[X_t=j, X_u=k, X_s=i]}{\mathbb{P}[X_s=i]} \cdot \frac{\mathbb{P}[X_u=k, X_s=i]}{\mathbb{P}[X_u=k, X_s=i]} \\
			&= \sum\limits_{k \in \mathcal{S}} \underbrace{\mathbb{P}[X_t=j | X_u=k, X_s=i]}_{\mathbb{P}[X_t=j | X_u=k]} \mathbb{P}[X_u=k | X_s=i] \\
			&= \sum\limits_{k \in \mathcal{S}} p_{ik}(s,u) p_{kj}(u,t)
		\end{align*}
		% use Baye's theorem for conditional probabilities: 
	\end{block}
	In the above we used $\mathbb{P} [A \cap B | C] = \mathbb{P}[A | B \cap C] \cdot \mathbb{P}[B | C]$ as well as the Markov porperty 
		as well as assuming that $\mathbb{P}[X_u=k, X_s=i] \neq 0$.
\end{frame}


\begin{frame}{Markov Model}
	To model a life Insurance we need three ingredients:
	\begin{itemize}
		\item a Markov chain $(X_t)_{t \in \mathbb{N}}$
		\item a one-year discount factor $v=\tfrac{1}{1+i}$
		\item contract functions $a_{i}^{pre}(t)$ and $a_{ij}^{post}(t)$
	\end{itemize}
	\vskip 0.5cm
	The starting point of an Markov model are the various possible conditions for an insured person, building the state space $\mathcal{S}$.
	E.g. $\mathcal{S} = \{ \text{'living'}, \text{'death'}\}$.
\end{frame}


\begin{frame}{Induced Cashflow \& Mathemtical Reserve}
	A central task in life insurance is the determination of the actuarial reserve, i.e.,
	the amount of money which has to be set aside at a given time t to be able to meet
	all future obligations/benefits towards each policy.\\
	We denote by $A_t$ the payments that are due for a policy at time t.
	$(A_t)_{t \in \mathbb{N}}$ is a stochastic process.
	\begin{equation*}
		A_t = a_{X_t}^{Pre}(t) + a_{X_{t-1} X_{t}}^{Post}(t)
	\end{equation*}
	where $a_{ij}^{Post}(-1)=0$ for all i,j $\in \mathcal{S}$, $t\in \mathbb{N}$.

\end{frame}


\begin{frame}{Induced Cashflow \& Mathemtical Reserve}
	We set $I_i(t) = \mathbbm{1}_{\{X_t = i\}}$. Then we can compute the {\bfseries induced cash flows} as follows:
	\begin{equation*}
		A(t) = \underbrace{\sum\limits_{i \in \mathcal{S}} I_i(t) \cdot a_{i}^{pre}(t)}_{\text{annuity}} + 
		\underbrace{\sum\limits_{i,j \in \mathcal{S}}  I_i(t) \cdot I_j(t+1) \cdot a_{ij}^{post}(t)}_{\text{captial/lump sum paid at time t+1}}
	\end{equation*}
	Idea: $A(t)$ are the payments are due at time t for a given policy. We can also compute the present value (PV) of A(t) which is given by:
	\begin{equation*}
		\tilde{A}(t) = \sum\limits_{i \in \mathcal{S}} I_i(t) \cdot a_{i}^{pre}(t) + v \cdot \sum\limits_{i,j \in \mathcal{S}}  I_i(t) \cdot I_j(t+1) \cdot a_{ij}^{post}(t)
	\end{equation*}
	Finally we can define the mathematical reserve at time t as:
	\begin{align*}
		V_j(t) = \mathbb{E}[\text{PV of future cash flows} | X_t=j] = \mathbb{E}[\sum\limits_{\tau=0}^{\infty} v ^\tau\tilde{A}(t + \tau) | X_t=j]
	\end{align*}
\end{frame}


\begin{frame}{Mathematical Reserve}
	We can compute the mathematical reserves with the following results:
	\begin{align*}
		\mathbb{E}[I_i(t+\tau) | X_t=j] &= p_{ji}(t,t+\tau) \\
		\mathbb{E}[I_i(t+\tau)\,I_k(t+\tau + 1) | X_t=j] &= p_{ji}(t,t+\tau) p_{ik}(t+\tau,t+\tau + 1)
	\end{align*}
	\vskip 0.5cm
	Hence the reserve is given as:
	\begin{align}
		V_j(t) &= \sum\limits_{\tau=0}^{\infty} v^\tau \left(\sum_{i \in \mathcal{S}} a_i^{Pre}(i+ \tau) p_{ji}(t, t+\tau)  \right.\nonumber\\
		&  + \left. v \sum_{i,k \in \mathcal{S}} a_{ik}^{Post}(t+ \tau) p_{ji}(t, t+\tau) p_{ik}(t+\tau,t+\tau + 1) \right)\\
		\label{eq:reserve}\nonumber
	\end{align}
	Hence $V_j(t)$ is the current value of the future actuarial reserve cash flow $(A)$ based on todays information, i.e. $X_t=j$.
\end{frame}


\begin{frame}{Thiele Equation}
	We can relate the mathematical reserves at two subsequent time points t and t+1 via the following equation:
	\begin{block}{Theorem (Thiele's difference equation)}
		The mathematical reserve between two subsequent periods are related by:
		\begin{equation*}
			V_j(t) = a_{j}^{pre}(t) + \sum_{i \in \mathcal{S}} v \cdot p_{ji}(t,t+1) \cdot (a_{ji}^{post}(t) + V_i(t+1))
		\end{equation*}
	\end{block}	
	{\bfseries Remarks:}\\
	\begin{itemize}
		\item Calculates the expected reserve directly using transition probabilities,
		 while simulation estimates it by averaging over many random trajectories
		\item To solve Thiele's equation we need the boundary condition $V_j(T) = 0$ for all $j \in \mathcal{S}$.
		\item The Thiele equation is leads to the same results as for the classical insurance model (using commutation functions)
		\item Forward computation of reserves is possible, too. but numerically less stable.
	\end{itemize}
\end{frame}

\begin{frame}{Proof Thiele Equation}
	% {\bfseries Proof.}
	We start the prove by separating the above sum into $\tau=0$ and the rest:\\
	Lets start with $\tau=0$:
	\begin{align*}
		&\sum_{i \in \mathcal{S}} a_i^{Pre}(t) \underbrace{p_{ji}(t,t)}_{\delta_{ij}} + v \sum_{i,k \in \mathcal{S}} a_{ik}^{Post}(t) p_{ji}(t,t)p_{ik}(t,t+1) \\
		&= a_j^{Pre}(t) + v \sum_{k \in \mathcal{S}} a_{jk}^{Post}(t) p_{jk}(t,t+1)
	\end{align*}
	Continute with $\tau \geq 1$ and using Chapman-Kolmogorov:
	\begin{align*}
		&\sum_{\tau \geq 1} v^\tau \left( \sum_i a_i^{Pre}(t+\tau)\cdot \underbrace{p_{ji}(t,t+\tau)}_{\sum_l p_{jl}(t, t+1)p_{li}(t+1,t+\tau)} \right.\\
		&+ \left. \sum_{i,k} a_{ik}^{Post}(t+\tau) \cdot\underbrace{p_{ji}(t,t+\tau)}_{\sum_l p_{jl}(t, t+1)p_{li}(t+1,t+\tau)} \cdot p_{ik}(t+\tau,t+\tau + 1) \right)
	\end{align*}
\end{frame}

\begin{frame}{Proof Thiele Equation}
	By factor out terms which are independent of $\tau$ as well as rearranging the $\tau$-sum ($\tau-1 \to \tau$) we get:
	\begin{align*}
		&\sum_l p_{jl}(t, t+1) \cdot v \cdot \left( \sum_{\tau \geq 0} v^\tau p_{li}(t+1, t+1+\tau) \right.\\
		&\left.\left[ \sum_i a_i^{Pre}(t+1+\tau) + \sum_{i,k} a_{ik}^{Post}(t+1+\tau)p_{ik}(t+1+\tau, t+\tau+1+1) \cdot v\right] \right)\\
	\end{align*}
	Comparing the expression in $()$ with equation (1) we recognize that this is just $V_l(t+1)$.
	Now combing both parts we get:
	\begin{equation*}
		V_j(t) = a_j^{Pre}(t) + v \sum_{k \in \mathcal{S}} p_{jk}(t,t+1) \left( a_{jk}^{Post}(t) + V_k(t+1) \right)
	\end{equation*}
	This concludes the proof.
\end{frame}


\begin{frame}{How to simulate a trajectory of a discrete time, finite State space Markov Chain}
	To simulate a trajectory of a discrete time, finite State space Markov Chain sith finite state space $\mathcal{S} = \{1,2,3, \dots , n\}$ and translation probability matrix $P = (p_{ij})$ we can use the {\bfseries Inverse Transformation Method}:
	\begin{enumerate}
		\item Start with initial state $X_0 = i_0$
		\item At each time step $t$, given $X_t = i$ \\ \begin{itemize}
																\item Generate $U \sim \text{Uniform}(0,1)$
																\item Find $j$ such that $\sum_{k=1}^{j-1} p_{ik} \leq U < \sum_{k=1}^{j} p_{ik}$
																\item Set $X_{t+1} = j$ 
														\end{itemize}
		
	\end{enumerate}
	This algortihm fulfills the Markove no-memory property by construction since only the current state $X_t$ is used to determine the next state $X_{t+1}$.
\end{frame}



\begin{frame}{Simulate mathematical reserve}
	The expression $V(X(\omega))[0]$ denotes the PV of future cash flows generated by $X(\omega)$ at time $t=t_0$ for a given trajectory of the Markov chain. Recall that the mathematical reserve is defined as $V_j(t) = \mathbb{E}[\text{PV of future cash flows} | X_{t_0}=j]$.\\
	Now given n independent trajectories $X(\omega_1), X(\omega_2), \ldots, X(\omega_n)$ we know that by the {\bfseries Law of Large Numbers}:
	\begin{equation*}
		\lim_{n\to\infty} \frac{1}{n}\sum_{k=1}^{n} V(\omega_k)[0] = V(t=t_0), 
	\end{equation*}
	i.e., the average of the PV of future chash flows over many independent trajectories converges to the true mathematical reserve at time $t=t_0$ almos surely.
\end{frame}


\begin{frame}{Determination of cumulative Probability Density of the reserves}
	Let $\{V(\omega_k)[0]\}_{k=1}^{n}$ be independent trajectoires. The empirical ccumulatice distribution function is given by: $$F_n(v) = \frac{1}{n}\sum_{k=1}^{n} \mathbbm{1}_{[-\infty, v]}(V(\omega_k)[0])$$
	\vskip 0.5cm
	Then define the random variable $D_n = \sup_{v \in \mathbb{R}} \left| F_n(v) - F(v) \right|$. By the {\bfseries Glivenko-Cantelli Theorem} we know that $D_n \xrightarrow{a.s.} 0$ for $n \to \infty$.\\
	\vskip 0.5cm
	Hence if we have enough samples/trajectories we can approximate the cumulative distribution function of the mathematical reserves.
\end{frame}



% \begin{frame}{Simulation of individual cash flows}
	
% \end{frame}
% \begin{frame}{Fixed Point Iteration Method}

%   \begin{block}{Definition}<1->
% A point p is a \textbf{fixed point} for a given function g if g(p)= p.
% \end{block}
%  \begin{block}{Remark}<2->
%  Fixed point problems and root finding problems are infact equivalent.
%  \begin{itemize}
% \item<1->if p is a fixed point of the function g, then p is a root of the function
% $$f(x)=[g(x)-x]h(x)$$ [as long as $h(x)\in \mathbb{R}$]
% \item<2->if p is a root of the function of f, then p is a fixed point of the function 
% $$g(x)=x-h(x)f(x)$$ [as long as $h(x)\in \mathbb{R}$]
% \end{itemize}
% \end{block}
%  \end{frame}

\section{Task 2 : Stopping to pay Premium}

\begin{frame}{Mixed Endowment Insurance}
	What is a mixed endowment?
	\begin{itemize}
		\item a mix of a term (temporary death) and a pure endowment insurance
		\item a lump sum is payable on death or reachinga certain age
	\end{itemize}
	Let us look at these random variables individually
	\begin{itemize}
		\item {\bfseries term insurance }: $Z_1 = v^{k+1} \mathbbm{1}_{k<n}$ \\
		\vskip 0.2cm
		$A_{x:\overline{n}|}^{\quad 1} = \mathbb{E}[Z_1] = \sum_{k=0}^{n-1} v^{k+1} \prescript{}{k}{p}_x \cdot q_{x+k} = \sum_{k=0}^{n-1} v^{k+1} \cdot\tfrac{l_{x+k}}{l_x}\cdot\tfrac{d_{x+k}}{l_{x+k}} $ \\ \vskip 0.2cm $= \sum_{k=0}^{n-1} \frac{C_{x+k}}{D_x}= \frac{M_x - M_{x+n}}{D_x}$
		% 						&= \sum_{k=0}^{n-1} \frac{C_{x+k}}{D_x} = \frac{M_x - M_{x+n}}{D_x}
		% \begin{align*}
		% 	A_{x:\overline{n}|} = \mathbb{E}[Z_1] &= \sum_{k=0}^{n-1} v^{k+1} \prescript{}{k}{p}_x \cdot q_{x+k} = \sum_{k=0}^{n-1} v^{k+1} \cdot\tfrac{l_{x+k}}{l_x}\cdot\tfrac{d_{x+k}}{l_{x+k}} \\
		% 						&= \sum_{k=0}^{n-1} \frac{C_{x+k}}{D_x} = \frac{M_x - M_{x+n}}{D_x}
		% \end{align*}
		\item {\bfseries pure endowment}: $Z_2 = v^{k+1} \mathbbm{1}_{k\geq n}$
		\item $ A_{x:\overline{n}|}^{1} = \mathbb{E}[Z_2] = \sum_{k=n}^{\infty} v^n \mathbb{P}[K=k] = v^n \mathbb{P}[K \geq n] = v^n \cdot \prescript{}{n}{p}_x = \frac{D_{x+n}}{D_x}$
		
		\item Thus the expectation value for the mixed endowment is given by \\
			$$ A_{x:\overline{n}|} = \mathbb{E}[Z] = \mathbb{E}[Z_1] + \mathbb{E}[Z_2] = \frac{M_x - M_{x+n} + D_{x+n}}{D_x}$$
	\end{itemize}
	% \begin{itemize}
	% 	\item Consider a mixed endowment $A_{80:\overline{10}|}$ maturing at age 90 with benefit insured $L=100'000$.
	% 	\item Techinical interest rate $i=2\%$ and Preimum is paid anually prenumerando
	% \end{itemize}
\end{frame}

\begin{frame}{Equivalence Principle}
	In order to determine the premium one typically uses the Equivalence Principle
	\begin{enumerate}
		\item $\mathbb{E}[L] = 0$ where $L$ denotes the loss, or equivalently,
		\item The expected value of premiums is equal to the expected value of benefits.
	\end{enumerate}
	Remark: \\
	\begin{itemize}
		\item The Equivalece Principle is {\bfseries equivalent} to the requirement of the mathematical reserve at inception to be zero.
	\end{itemize}
	As an example consider a term insurance where we define the loss as
	$$L = C \cdot v^{k+1} \mathbbm{1}_{k<n} - \Pi \cdot \ddot{a}_{\min (k,n)}$$
	By the Equivalence Pricinple we have $\mathbb{E}[L] = 0$ which leads to $$ \Pi\cdot \ddot{a}_{x:\overline{n}|} = C \cdot A_{x:\overline{n}|}^{\quad 1}.$$
\end{frame}


\begin{frame}{Mathemical Reserves}
	The Mathematical Reserves V at given time $t$ are defined as \\  
	{\bfseries V = PV(future benefits) - PV(future premiums)}
	\vskip 0.5cm
	For a mixed endowment insurance with n annual premiums $\Pi$ the mathematical reserve at time t is given by (expressed in commutation functions)
	\begin{equation*}
		\prescript{}{t}{V}_x = \frac{M_{x+t} - M_{x+n} + D_{x+n} - \Pi\cdot (N_{x+t} - N_{x+n})}{D_{x+t}}
	\end{equation*}
	
\end{frame}


\begin{frame}{2.1 Premium for Product at Inception}
	\begin{itemize}
		\item Compute the premium by means of the equivalence principle.
		
		\item For the mixes endowment with benefit $L$ we have
				\begin{equation*}
					\Pi = L \frac{A_{x:\overline{n}|}}{\ddot{a}_{x:\overline{n}|}} = L \left(\frac{M_x - M_{x+n} + D_{x+n}}{N_x - N_{x+n}}\right)
				\end{equation*}
	\item Plugging in the values we get $\Pi \approx 12'302.98$
	\end{itemize}
\end{frame}


\begin{frame}{2.2 Benefit $\tilde{L}$ after policyholder stops after one premium. }
	\begin{itemize}
		\item Combining the formulas for the mixed endowment insurance $A_{x+k : \overline{n-k}|}$ and the mathematical reserve $\prescript{}{k}{V}_x$
			we obtain the following expression for the reduced benefit using commutation functions
			\begin{equation*}
					\prescript{}{1}{\tilde{L}} = \frac{\prescript{}{k}{V}_{x}}{A_{x+k:\overline{n-k}|}} = \prescript{}{k}{V}_x \cdot \frac{D_{x+k}}{M_{x+k} - M_{x+n} + D_{x+n}}
			\end{equation*}
		\item Result of the python code is: $\prescript{}{1}{\tilde{L}} \approx 9228.77$.
	\end{itemize}
	
\end{frame}


\begin{frame}{2.3 Benefit Level $\tilde{L}$ as a function of the number of paid premiums}
	\begin{itemize}
		\item Let $k$ be the number of paid premiums then by the previous exercise
			\begin{equation*}
				\prescript{}{k}{\tilde{L}} = \frac{\prescript{}{k}{V}_{x}}{A_{x+k:\overline{n-k}|}} = \prescript{}{k}{V}_x \cdot \frac{D_{x+k}}{M_{x+k} - M_{x+n} + D_{x+n}}
			\end{equation*}
		\begin{table}[h]
  			\begin{tabular}{|c|c|c|}
    		\hline
    		$k$ & Reserve  $\prescript{}{j}{V}_x$ &  $\text{Benefit Level }\prescript{}{k}{\tilde{L}}$ \\
    		\hline
    		1 & 8062.41 & 9228.77 \\
    		2 & 16260.21 & 18375.48 \\
			3 & 24650.21 & 27498.51 \\
			4 & 33308.23 & 36670.15 \\
			5 & 42335.99 & 45980.83\\
			6 & 51870.01 & 55545.00 \\
			7 & 62095.67 & 65509.06 \\
			8 & 73266.94 & 76062.14 \\
			9 & 85736.24 & 87450.96 \\
			10 & 100000.0 & 100000.0\\
			\hline
  			\end{tabular}
			\end{table}
	\end{itemize}	
\end{frame}


\begin{frame}{2.4 Which equivalence principle is fulfilled for the first premium assuming only one premium is paid.}
	\begin{itemize}
		\item $t=0$ and premium $P$ 
		\item Payout/Benefits:\\ i.) if $t \in [0,1)$ then $L=100'000$ ii.) if $t \in [1,10)$ then $\prescript{}{t}{\tilde{L}}$
		\item by the {\bfseries Equivalence Principle}: $\mathbb{E}(premium) = \mathbb{E}(benefits)$
		\item the benefits are computed as
			\begin{align*} 
				\mathbb{E}(benefits) &= v \cdot q_x \cdot L + v^2 \cdot \prescript{}{1}{\tilde{L}} \cdot p_x \cdot q_{x+1} + \dots + v^{10} \cdot \prescript{}{1}{\tilde{L}} \cdot \prescript{}{10}{p}_{x}\\
												&= v \cdot q_x \cdot L + v \cdot p_x \cdot \prescript{}{1}{\tilde{L}}\cdot A_{x+1 : \overline{9}|}
			\end{align*}
		\item A short check in the jupyter notebook shows that $\mathbb{E}(premium) = \mathbb{E}(benefits)$
	\end{itemize}
\end{frame}


% \begin{frame}{2.5 Which equivalence principle is fulfilled for the second premium assuming only one premium is paid.}
% 	\begin{itemize}
% 		\item t=1
% 		\item Assuming one premium is paid and compute the second one
% 		\item Assuming that insured person survives the first year
% 		\item Insurer is at risk to pay $L$ if the death occurs in $[1,2)$
% 		\item From time $t=2$ on until maturity the insruer is at risk of $\prescript{}{2}{\tilde{L}}_x$
% 		\item By the Equivalence Principle $\mathbb{E}(premium) = \mathbb{E}(benefits)$
% 		\item $\mathbb{E}(premium) = P + \prescript{}{1}{V}_x$
% 	\end{itemize}
% \end{frame}

% \begin{frame}{2.5 Which equivalence principle is fulfilled for the second premium assuming only one premium is paid.}
% 	\begin{itemize}
% 		\item 	Similar to before we compute the benefits:
% 				\begin{align*}
% 					\mathbb{E}(benefits) &= v \cdot q_{x+1} \cdot L + v^2 \cdot \prescript{}{2}{\tilde{L}} \cdot p_{x+1} \cdot q_{x+2} + \dots + v^{9} \cdot \prescript{}{2}{\tilde{L}} \cdot \prescript{}{9}{p}_{x}\\
% 					&=v \cdot q_{x+1} \cdot L + v \cdot p_{x+1} \cdot \prescript{}{2}{\tilde{L}} \cdot A_{x+2 : \overline{8}|}
% 				\end{align*}

% 		\item A short check in the jupyter notebook shows that $\mathbb{E}(premium) = \mathbb{E}(benefits).$
% 	\end{itemize}
	
% \end{frame}
\begin{frame}{2.5 Which equivalence principle is fulfilled for the second premium assuming only one premium is paid.}
	\begin{itemize}
		\item Consider $t=1$
		\item Assuming one premium is paid at $t=0$
		\item Assuming that insured person survives the first year
		\item Insurer is at risk to pay $L$ if the death occurs in $[0,1)$
		\item From time $t=1$ on until maturity the insruer is at risk of $\prescript{}{1}{\tilde{L}}_x$
		\item By the Equivalence Principle $\mathbb{E}(premium) = \mathbb{E}(benefits)$
		\item $\mathbb{E}(premium) = \prescript{}{1}{V}_x$
	\end{itemize}
\end{frame}

\begin{frame}{2.5 Which equivalence principle is fulfilled for the second premium assuming only one premium is paid.}
	\begin{itemize}
		\item 	Similar to before we compute the benefits:
				\begin{align*}
					\mathbb{E}(benefits) &= v \cdot q_{x+1} \cdot \prescript{}{1}{\tilde{L}} + v^2 \cdot \prescript{}{1}{\tilde{L}} \cdot p_{x+1} \cdot q_{x+2} + \dots + v^{9} \cdot \prescript{}{1}{\tilde{L}} \cdot \prescript{}{9}{p}_{x}\\
					&= \prescript{}{1}{\tilde{L}} \cdot A_{x+1 : \overline{9}|}
				\end{align*}

		\item A short check in the jupyter notebook shows that $\mathbb{E}(premium) = \mathbb{E}(benefits)$ as well as the equality is exactly the definition for the reduced benefit $\prescript{}{1}{\tilde{L}}$.
	\end{itemize}
	
\end{frame}





\section{Task 3: Disability Insurance on two lives}

% \begin{frame}{Problem}
% 	To sample Markov chains install the following python package to guarantee functionality of the class \texttt{Markov}: \href{https://libraries.io/pypi/markovlv/2.6.0}{markovlv}.
% \end{frame}

\begin{frame}{3.1 Uncorrelated Vs. independent}
	Two random variables $X$ and $Y$ are \textbf{stochastically independent} if and only if their joint probability distribution factorizes:
	\begin{equation*}
		P(X = x, Y = y) = P(X = x) \cdot P(Y = y) \quad \forall x, y
	\end{equation*}
	Or equivalently:
	\begin{equation*}
		P(Y = y \mid X = x) = P(Y = y) \quad \forall x, y.
	\end{equation*}
	\textit{Intuitively:} Knowing $X$ provides no information about $Y$\\
	\vskip 0.3cm
	Two random variables $X$ and $Y$ are \textbf{uncorrelated} if their covariance is zero: $$
	\text{Cov}(X, Y) = \mathbb{E}[(X - \mathbb{E}[X])(Y - \mathbb{E}[Y])] = 0.$$
	Or equivalently: 
	\begin{equation*}
	\quad \mathbb{E}[XY] = \mathbb{E}[X] \cdot \mathbb{E}[Y].
	\end{equation*}
	% \textit{In other words} there is no \textbf{linear} relationship between $X$ and $Y$.

	\end{frame}

	\begin{frame}{3.1 Uncorrelated Vs. Independent}
		\begin{block}{Independence $\Rightarrow$ Uncorrelated, Proof}
		Let $X$ and $Y$ be discrete and independent random variables.
	\begin{align*}
		\mathbb{E}[XY] &= \sum_x \sum_y xy \cdot P(X=x, Y=y) \\
						&= \sum_x \sum_y xy \cdot P(X=x) \cdot P(Y=y) = \mathbb{E}[X] \cdot \mathbb{E}[Y].
	\end{align*}
	This concludes the proof.
	\end{block}

	\vspace{0.5cm}

	\begin{alertblock}{Uncorrelated $\not\Rightarrow$ Independence}
	Two variables can be uncorrelated yet still be dependent!
	\end{alertblock}
\end{frame}


\begin{frame}{3.1 Counterexample: Uncorrelated but Dependent}
% \frametitle{Uncorrelated but Dependent}

Let $X \sim \text{Uniform}(-1, 1)$ and $Y = X^2$

\vspace{0.3cm}

\begin{itemize}
    \item $\mathbb{E}[XY] = \mathbb{E}[X^3] = 0$ (by symmetry)
    \vspace{0.2cm}
    \item $\mathbb{E}[X] = 0$ and $\mathbb{E}[Y] = \frac{1}{3}$
    \vspace{0.2cm}
    \item Therefore: $\mathbb{E}[X] \cdot \mathbb{E}[Y] = 0$
    \vspace{0.2cm}
    \item So $\text{Cov}(X,Y) = 0$ $\Rightarrow$ \textbf{uncorrelated}
    \vspace{0.2cm}
    \item But $Y$ is completely determined by $X$ $\Rightarrow$ \textbf{dependent!}
\end{itemize}

\vspace{0.5cm}

% \textbf{Key insight:} Independence means \textit{no relationship whatsoever}, while being uncorrelated only rules out \textit{linear} relationships.

\end{frame}


% \begin{frame}{3.3}

% \end{frame}


\begin{frame}{3.4 MC Averages Analysis}
	Here we show how the Monte Carlo estimates move with increasing number of simulated trajectories.
	We see that with increasing number of simulations the estimates converge to the true mathematical reserve (computed via Thiele's equation).
	\begin{figure}
		\includegraphics[width=0.9\textwidth]{Plots/Mean_Reserve_Estimates_with_SEM_Markov_States_34.png}
	\end{figure}
\end{frame}



\begin{frame}{3.4 CDF and Quantiles for 100'000 Simulations}
	% \begin{figure}
	% 	\includegraphics[width=0.7\textwidth]{Plots/Empirical_CDF_Markov_States_34.png}
	% \end{figure}
	% \begin{figure}
	\begin{figure}
		\begin{minipage}{0.49\textwidth}
			\includegraphics[width=\textwidth]{Plots/Empirical_CDF_Markov_States_34.png}
			\caption{Empirical CDF of the reserve}
		\end{minipage}
		\hfill
		\begin{minipage}{0.49\textwidth}
			\includegraphics[width=\textwidth]{Plots/Quantiles_Markov_States_34.png}
			\caption{Quantiles of the reserve}
		\end{minipage}
	\end{figure}

\end{frame}

\begin{frame}{3.5 coefficient of variation across different states}	
	\begin{figure}
		\includegraphics[width=0.85\textwidth]{Plots/Coefficient_of_Variation_Markov_States_35.png}
	\end{figure}
\end{frame}


\begin{frame}{3.5 Discussion of Risk Levels of States}
	In order to dicuss the riskiness of the states we consider two risk meaures: value at risk (VaR) and the coefficient of variation (CoV):
	\begin{itemize}
		% \item The {\bfseries VaR} at level provides a concrete monetary amount for tail losses, making it ideal for determining how much capital to hold in reserve \\
		\item The {\bfseries VaR} gives the maximum loss that will not be exceeded with a specified confidence level over a given time period
		\item Directly answers: "What is the maximum loss we expect not to exceed in $95\%$ of cases?"
		\vskip 0.3cm
		\item The {\bfseries CoV} normalizes volatility by the mean, allowing fair comparison between states with very different expected losses \\
		\item Directly answers: "Which state has the most unpredictable or volatile losses relative to its average?"
	\end{itemize}
	% In terms of the insurer's risk management, VaR helps ensure sufficient capital buffers, while CoV aids in portfolio diversification by identifying states with high relative volatility.\\
	\begin{itemize}
		\item W.r.t VaR at level $95\%$ we conlcude that the state $(3,3)$ and $(D,D)$ are the most dangerous states.
		\item W.r.t CoV the most dangerous state is $(D,T)$.
	\end{itemize}
\end{frame}

\begin{frame}{3.5 Discussion of Risk Levels of States}
	Now since it is stated in the task to assume that the insurer has the adequate amount of reserves we assume that the VaR is also ready covered since this measure tells us on how much to reserve.\\
	Hence we judge the riskiness by the CoV, which measures how much harder it is to maintain adequate reserves over time.
	\begin{itemize}
		\item The state $(D,T)$ is the most dangerous state (has highest CoV) since the man is disabled and rather young (30y).
			 Hence he is has three possibilities: reactivate, remain disabled for extended periods.
		\item The diversification from having two people in similar but independent states reduces relative variability
		\item The state $(3,3)$ is the safest one (although highest annuity to be paid) since both persons are disabled and either remain disabled or die. 
		\item So ranking the states according to their risk level (from riskiest to safest) is given by sorting the CoV in decreasing order:
			\begin{align*}
				(D,T) &> (S,S) >  (D,S) > (T,D) > (S,D) > (D,D) \\
				&> (S,3) > (3,T) > (3,S) > (T,3) > (3,3)
			\end{align*}
	\end{itemize}	
\end{frame}

\begin{frame}{3.6 Dicussion of Changes in Risk when changing from independent to Dependent Random Variables }
	We make the random variables dependent by adapting the follwoing in the code:
	\begin{align*}
		&\texttt{a.vSetLevelsMort(dMortSS = 1., dMortSD =10., dMortDD=1., dMortST=1., dMortDT=1.)}\\
		&\texttt{a.vSetLevelsDis(dDisSS=1., dDisSD=1., dDisST=1.)}\\
		&\texttt{a.vSetLevelsReact(dReactSD=1., dReactDD=1., dReactDT=1.)}
	\end{align*}
This means that if one person is disabled and the other is healthy, the probability that one of them will die is ten times higher.
\end{frame}

\begin{frame}{3.6 Dicussion of Changes in Risk when changing from independent to Dependent Random Variables }
	\begin{figure}
		\includegraphics[width=1.\textwidth]{Plots/Empirical_CDF_Comparison_Independent_Dependent_Markov_States_36.png}
	\end{figure}
\end{frame}


\begin{frame}{3.6 Dicussion of Changes in Risk when changing from independent to Dependent Random Variables }
	\begin{figure}
		\includegraphics[width=0.8\textwidth]{Plots/Coefficient_of_Variation_Markov_States_36.png}
	\end{figure}
\end{frame}


\begin{frame}{3.6 Dicussion of Changes in Risk when changing from independent to Dependent Random Variables }
	\begin{itemize}
		\item The CDFs for the dependent case with possible intermediate SD state are shifted to the left compared to the independent case.
		\item This is intuitive since the higher chance of death leads to lower expected cash flows to be paid out by the insurer.
		\item Hence the VaR at level $95 \%$ is reduced in the dependent case meaning that the insurer has to reserve less capital.
		\item In terms of the CoV we observe that for policies with an intital SD configuration, the CoV is increased in the dependent case.
		This is due to the following fact : $\text{Var}(X + Y) =  \text{Var}(X) + \text{Var}(Y) + 2\text{Cov}(X,Y)$.
		\item In the case when making the random variables dependent the covariance term is {\bfseries positive} leading to higher overall variance and hence higher CoV.
		\item The considered model for dependence is a special case, because we only changed one weighting factor. If we change more parameters the results might differ and can be less intuitive.
	\end{itemize}
\end{frame}


\begin{frame}{3.7 Cashflows CDF distribution for age 65 and 66}
	\begin{figure}
		\includegraphics[width=1.\textwidth]{Plots/Empirical_CDF_Cashflows_All_States.png}
	\end{figure}
\end{frame}


\begin{frame}{3.7 Observations}
	\begin{itemize}
		\item A trend we observe over all states is that the CDF for age 66 is shifted below the CDF for age 65.
		\item This is intuitive since the older the person the higher the probability of death or also getting disabled and the quantiles are smaller at a certain level.
		\item Example: For the initial state  (3,S) and a cash flow of 12,000, the probability at age 66 is lower than at age 65, indicating that the insurer has to pay this amount less often but instead needs to pay a higher amount.
		
	\end{itemize}
\end{frame}



\begin{frame}{3.8: 8-policy portfolio simulation}
	\begin{itemize}
		\item Assuming the 8 policies are independent, we compute the total loss as the sum of the indiviudal mathematical reserves.
			$$ \mathbb{E}[L] = \sum_{i=1}^8 \mathbb{E}[L_i] = 1'816'458.10$$
		\item Now we want to compute the stop loss premium $\Pi(\beta)$ for the { \bfseries deductibles} $\beta \in \{1.5\cdot \mathbb{E}[L], 3 \cdot \mathbb{E}[L]\}$
		\item The stop loss premium is defined as $$\Pi(\beta) = \mathbb{E}[(L- \beta)_+] =  \int_{\beta}^{\infty} (1-F(x)) \, dx $$
		\item In order to create a histogram-based approximation of the cumulative distribution function $F(x)$ we use 1'000'000 simulated portfolio losses (in total 8'000'000 simulations) 
	\end{itemize}
\end{frame}


\begin{frame}{3.8: 8-policy portfolio simulation}
	Here we plot the histogram and empirical CDF of the portfolio reserves for 1'000'000 siumlation runs per policy (in total 8'000'000 simulations).
	\begin{figure}
		\includegraphics[width=1.\textwidth]{Plots/Histogram_and_Empirical_CDF_Portfolio_Reserves.png}
	\end{figure}
	The above histogram contains of $\approx 4800$ bins of width $500$.
\end{frame}


\begin{frame}{3.8: 8-policy portfolio simulation}
	Now we shortly describe how to compute the stop loss premium $\Pi(\beta)$ using a histogram-based approximation of the cumulative distribution function $F(x)$.
	\begin{itemize}
		\item Let the bins of the histogram be $[a_1,a_2), \,[a_2,a_3), \ \dots,\,[a_N,a_{N+1})$.
		\item Assume the survival function is constant on each bin: $\bar F(x) = 1 - F(x) \approx \bar F_i $ where $\bar F_i$ denotes the value of the survival function on bin $i$.
		\item The integral representation approximated by the Riemann sum: $$\Pi(\beta) \approx \sum_{i:\, a_i \ge \beta} \bar F_i \,(a_{i+1}-a_i) $$
		\item Plugging in the values for $\beta$ we get:
			\begin{align*}
				\Pi(1.5\cdot \mathbb{E}[L]) &= 8.47\\
				\Pi(3\cdot \mathbb{E}[L]) &= 0.0
			\end{align*}
	\end{itemize}
\end{frame}


\section{Appendix}

\begin{frame}{Commutation Functions}
	We have the following declarations: $l_x$ : number of living persons at age $x$ \\
	$d_x = l_x - l_{x+1}$ : number of deaths between age $x$ and $x+1$
	\begin{align*}
		D_x &= v^x\cdot l_x, \quad N_x = \sum_{k=x}^{\infty} D_k,  \quad S_x = \sum_{k=x}^{\infty} N_k \\
		C_x &= v^{x+1} \cdot d_x, \quad M_x = \sum_{k=x}^{\infty} C_k, \quad R_x = \sum_{k=x}^{\infty} M_k
	\end{align*}
\end{frame}


\begin{frame}{More Formulas}
	Life Insurance with return beneft: in case of death before age $n$ the assets held up to that point in time paid-in premiums are returned
	\begin{align*}
		P_x \ddot{a}_{x:\overline{n}|} = P_x (IA)_{x:\overline{n}|}^1  + A_{x:\overline{n}|}^{\quad 1}
	\end{align*}
	In the case of old-age annuity:
	$$ P_x = \frac{\prescript{}{n}{|}\ddot{a}_x^{(m)}}{\ddot{a}_{x:\overline{n}|} - (IA)_{x:\overline{n}|}^1}$$
	where $(IA)_{x:\overline{n}|}^1$ denotes the linear increased death benefit: 
$$ (IA)_{x:\overline{n}|}^1 = n\cdot A_{x:\overline{n}|}^1 - A_{x:\overline{n-1}|} - \dots - A_{x:\overline{1}|} = \frac{R_x - R_{x+n} - n \cdot M_{x+n}}{D_x}$$ 
\end{frame}



\begin{frame}{}
	
\end{frame}




\end{document}